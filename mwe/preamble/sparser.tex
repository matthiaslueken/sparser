%required packages
\usepackage{xparse}
\usepackage{xstring}
\usepackage{xcolor}
\usepackage{datatool}
\usepackage{booktabs}
\usepackage{tabularx}
\usepackage{colortbl}
\usepackage{multirow}

%colors
% can be overwritten in main document
\makeatletter
\@ifundefinedcolor{sporange}{\definecolor{sporange}{HTML}{ff9c00}}
\makeatother
\makeatletter
\@ifundefinedcolor{spdarkgreen}{\definecolor{spdarkgreen}{HTML}{0CAC64}}
\makeatother
\makeatletter
\@ifundefinedcolor{spgray}{\definecolor{spgray}{HTML}{869191}}
\makeatother
\makeatletter
\@ifundefinedcolor{spred}{\definecolor{spred}{HTML}{FF0A33}}
\makeatother

% outputting numbers is done via the \num command from the siunitx package, but using it is optional
\ifcsname num\endcsname \else\newcommand{\num}{\relax}\fi

%%%%%%%%%%%%%%%%%%%%%%%%%%%%%
% Copyright (c) 2021 Matthias Lüken
% https://github.com/matthiaslueken/sparser/
%
% CONTENTS
%
% ------------------------------------------------------------------------------------------------------------------------------------------------------------------------------------------------------------
% Name						Use				Description													Parameters
% ------------------------------------------------------------------------------------------------------------------------------------------------------------------------------------------------------------
% \sparser_debug_out:n 		internal			outputs text when debug flag is set							text (string)
% ------------------------------------------------------------------------------------------------------------------------------------------------------------------------------------------------------------
% \sparse						Document		Parser for time and io statistics								file (string), debug out (0/1)
% \sparser_readdata:nn			internal			loops through lines of input file								file (string), debug out (0/1)
% \sparser_fsm:nn				internal			finite state machine of parser								state (string), active (0/1)
% \sparser_writedata:n 			internal			buffers and writes data to files								line from input file (string)
% \sparser_data_getdigits:n	internal			extracts digits from elements of clist							clist
% \sparser_file_time:nnn 		internal			generates file name for time output							folder, filename, extension (string)
% \sparser_file_io:nnn 			internal			generates file name for io output							folder, filename, extension (string)
% \sparser_close_stream:n 		internal			closes output stream and sets flag							output (string)			
% \sparser_open_stream:n 		internal			opens output stream and sets flag							output (string)			
% ------------------------------------------------------------------------------------------------------------------------------------------------------------------------------------------------------------
% \sparsertime					Document		generates a table with time statistics	file (string)				file (string)
% \sparserio	 				Document		wrapper function for io tables								fileA (string), fileB (string), skip (0/1), coloring (0/1)
% \sparseriosingletable 			internal			generates a table with io statistics							file (string)
% \sparseriotablecompare 		internal			generates a table comparing two io statistics					fileA (string), fileB (string), skip (0/1), coloring (0/1)
% \sparser_match_tables:nn	internal			adds rows to two dbs to make their first columns identical	databaseA (string), databaseB (string)
% \sparser_breakanywhere:Nn	internal			allows latex to wrap word anywhere							string (string), length (int)
% \sparser_colorscale:nnnnn	internal			adds color for table cell based on |#2-#1|/max(#3,#4)		a, b, x, y (integer)
% \sparser_dict:n 				internal			dictionary for multilingual support							phrase (string)
% ------------------------------------------------------------------------------------------------------------------------------------------------------------------------------------------------------------
%%%%%%%%%%%%%%%%%%%%%%%%%%%%%

\ExplSyntaxOn

%%%%%%%%%%%%%%%%%%%%%%%%%%%%%
% \sparser_debug_out:n
%%%%%%%%%%%%%%%%%%%%%%%%%%%%%

\cs_new:Npn \sparser_debug_out:n #1 {
	\bool_if:NTF \l__sparser_debug_bool {#1}{}
}

%%%%%%%%%%%%%%%%%%%%%%%%%%%%%
% \sparse
%%%%%%%%%%%%%%%%%%%%%%%%%%%%%

\NewDocumentCommand{\sparse}{m O{0}}
{
	\texttt{
		\sparser_readdata:nn { #1 }{ #2 }
	}
}

% variables for sparse

% file io streams
\ior_new:N \g__sparser_read_ior
\iow_new:N \g__sparser_write_time_iow
\iow_new:N \g__sparser_write_io_iow

% flags if write stream is open
\bool_new:N \l__sparser_output_time_open_bool
\bool_new:N \l__sparser_output_io_open_bool

%file names
\str_new:N \l__sparser_inputfile_str
\str_new:N \l__sparser_output_time_str
\str_new:N \l__sparser_output_io_str

% line
\tl_new:N \l__sparser_line_tl
\str_new:N \l__sparser_line_str

%data
\clist_new:N \l__sparser_data_comp_clist
\clist_new:N \l__sparser_data_exec_clist
\clist_new:N \l__sparser_data_time_clist
\clist_new:N \l__sparser_data_io_clist
\tl_new:N \l__sparser_tablename_tl

% internal state and active variable
\str_new:N \l__sparser_state_str
\bool_new:N \l__sparser_active_bool
\bool_new:N \l__sparser_ignore_bool
\bool_new:N \l__sparser_debug_bool

% batch and statement counters
\int_new:N \l__sparser_batch_int
\int_new:N \l__sparser_statement_int

%%%%%%%%%%%%%%%%%%%%%%%%%%%%%
% \sparser_readdata:nn
%%%%%%%%%%%%%%%%%%%%%%%%%%%%%

\cs_new:Npn \sparser_readdata:nn #1 #2
% parameter: #1: file of statistics text file
% parameter: #2: debug out flag
{
%switch on/off debug
	\int_case:nn #2
	{
		{0} {\bool_set_false:N \l__sparser_debug_bool}
		{1} {\bool_set_true:N \l__sparser_debug_bool}
	}

%	init state, batch and statement counters
	\str_set:Nn \l__sparser_state_str {init}
	\bool_set_false:N \l__sparser_active_bool
	\int_set:Nn \l__sparser_batch_int {0}
	\int_set:Nn \l__sparser_statement_int {0}	
	
%	output streams closed
	\bool_set_false:N \l__sparser_output_time_open_bool
	\bool_set_false:N \l__sparser_output_io_open_bool

%	store input file name and open stream
	\str_set:Nn \l__sparser_inputfile_str {#1}
	\ior_open:Nn \g__sparser_read_ior {\l__sparser_inputfile_str}
	
	\sparser_debug_out:n {parsing\ \textcolor{spdarkgreen}{#1}...\\}
%	loop through input file line by line
	\ior_map_inline:Nn \g__sparser_read_ior
		{
%			store line, remove all numbers and spaces, convert it to a string und take the last 105 characters
			\tl_set:Nn \l__sparser_line_tl {##1}
			\regex_replace_all:nnN {\s*|[0-9][0-9]*} {} \l__sparser_line_tl
			\str_set:Nx \l__sparser_line_str {\tl_to_str:N \l__sparser_line_tl}
			\str_set:Nx \l__sparser_line_str {\str_range:Nnn \l__sparser_line_str {-105} {-1}}

%			switch states if the line matches a key phrase
%			the machine stays in the same state untill a new key phrase is recognized,
%			but deactivates upon unknown input
%			key phrases are recognized for english and german statistics
			\str_case_e:nnTF \l__sparser_line_str
				{
%					compile time header
					{SQL Server parse and compile time:} {\sparser_fsm:nn {comp}{0}}
					{SQL Server-Analyse- und Kompilierzeit:} {\sparser_fsm:nn {comp}{0}}
%					execution time header
					{SQL Server Execution Times:} {\sparser_fsm:nn {exec}{0}}
					{SQL Server-Ausführungszeiten:} {\sparser_fsm:nn {exec}{0}}
%					io statistics
					{Scan count , logical reads , physical reads , read-ahead reads , lob logical reads , lob physical reads , lob read-ahead reads.} {\sparser_fsm:nn {io}{1}}
					{e , Read-Ahead-Lesevorgänge , logische LOB-Lesevorgänge , physische LOB-Lesevorgänge , Read-Ahead-LOB-Lesevorgänge.} {\sparser_fsm:nn {io}{1}}
%					rows affected
					{( rows affected)} {\sparser_fsm:nn {rows}{1}}
					{( Zeilen betroffen)} {\sparser_fsm:nn {rows}{1}}
					{( Zeile betroffen)} {\sparser_fsm:nn {rows}{1}}
%					compile or execution time data
					{, CPU-Zeit = ms, verstrichene Zeit = ms.} {\sparser_fsm:nn {}{1}}
					{CPU time = ms, elapsed time = ms.} {\sparser_fsm:nn {}{1}}
				}
				{
%					when matched and machine active, handle data
					\bool_if:NTF \l__sparser_active_bool
						{
							\sparser_debug_out:n {\textcolor{spgray}{data:\ }}
							\sparser_writedata:n {##1}
						}
						{
							\sparser_debug_out:n {\textcolor{spgray}{\str_range:Nnn {##1} {1}{32}}\\}
						}
				}
				{
%					when not matched
					\sparser_fsm:nn {unknown}{0}
					\sparser_debug_out:n {
						\textcolor{spred}{no\ match:\ }
						\textcolor{spgray}{\str_range:Nnn \l__sparser_line_tl {1}{32}}\\
					}
				}
		}

%	close input file streamf
	\ior_close:N \g__sparser_read_ior

%	close io output file stream	 if open
	\sparser_close_stream:n {io}

%	close comp output file stream if open
	\sparser_close_stream:n {time}
\sparser_debug_out:n {
	...finished\ parsing\ \textcolor{spdarkgreen}{#1}\\
}
}

%%%%%%%%%%%%%%%%%%%%%%%%%%%%%
% \sparser_fsm:nn
%%%%%%%%%%%%%%%%%%%%%%%%%%%%%

% upon switching to a new state, this function creates files, opens and closes streams and sets counters
\cs_new:Npn \sparser_fsm:nn #1 #2
% parameter: #1: new state, #2: active
{
% new states are not ignored by default. Set true if something is wrong and input shoud be ignored
	\bool_set_false:N \l__sparser_ignore_bool
	
%	actions depending on the new state
	\str_case_e:nnTF {#1}
	{
%		general layout: check previous state, set counters, create files

%		compile time
		{comp} {
%			compile times always mark the start of a new batch
%			set counters
			\int_set:Nn \l__sparser_batch_int {\l__sparser_batch_int+1}
			\int_set:Nn \l__sparser_statement_int {0}
%			check if time output stream is open
			\bool_if:NTF {\l__sparser_output_time_open_bool}
				{}%yes
				{%no: create new file
					\sparser_open_stream:n {time}
				}
		}
%		execution time
		{exec} {
%			check previous state
			\str_case_e:nnTF \l__sparser_state_str 
			{
				{io} {}%io followed by exec => same statement
				{init} {
					%init should not be followed by exed => ignore input
					\bool_set_true:N \l__sparser_ignore_bool
				}
			}
			{
				%match: do nothing
			}
			{
%				no match => new statement: set counters
				\int_set:Nn \l__sparser_statement_int {\l__sparser_statement_int+1}
			}
		}
		{io} {
%			check previous state	
			\str_case_e:nnTF \l__sparser_state_str
			{
				{init} {
%					io first state => no statistics time: set counters
					\int_set:Nn \l__sparser_batch_int {1}
					\int_set:Nn \l__sparser_statement_int {\l__sparser_statement_int+1}
				}
				{rows} {}%rows followes by io => same statement
				{io} {
%						if io was the last state, check if machine was active
						\bool_if:NTF \l__sparser_active_bool
							{}%yes: part of the same statement
							{%no: no count and statistics time off: new statement
								\int_set:Nn \l__sparser_statement_int {\l__sparser_statement_int+1}
							}
				}
			}
			{}
			{
%				no match => new statement: set counters
				\int_set:Nn \l__sparser_statement_int {\l__sparser_statement_int+1}
			}

%			close and open a new io file unless the last state was io and the machine was active				
			\bool_if:nTF {!(\str_if_eq_p:NN \l__sparser_state_str {io} && \l__sparser_active_bool)}
			{
				\sparser_close_stream:n {io}
				\sparser_open_stream:n {io}
			}
			{}
		}
		{rows} {
			\str_case_e:nnTF \l__sparser_state_str
			{
				{init} {}
			}
			{
				%first state menas no time statistics: ignore
				\bool_set_true:N \l__sparser_ignore_bool	
			}
			{		
%				check if exec list is empty
				\clist_if_empty:NTF \l__sparser_data_exec_clist
				{
%					empty: everythink ok, increase statement counter
					\int_set:Nn \l__sparser_statement_int {\l__sparser_statement_int+1}
				}
				{
%					not empty: something is wrong, dont change state and deactivate
					\bool_set_true:N \l__sparser_ignore_bool	
				}
			}
		}
	}
	{
%	match: set new state if not ignored
		\bool_if:nTF \l__sparser_ignore_bool
		{}
		{
			\str_set:Nn \l__sparser_state_str {#1}
		}
	}
	{}

%	(de)activate if not ignored
	\bool_if:nTF \l__sparser_ignore_bool
	{
		\bool_set_false:N \l__sparser_active_bool
	}
	{
		\int_case:nn #2
		{
			{0} {\bool_set_false:N \l__sparser_active_bool}
			{1} {\bool_set_true:N \l__sparser_active_bool}
		}
	}
	\sparser_debug_out:n {
		b: \int_use:N \l__sparser_batch_int\ 
		s: \int_use:N \l__sparser_statement_int\ 
		a: \bool_if:NTF \l__sparser_active_bool {1}{0}\ 
		state:\textcolor{spdarkgreen}{\str_range:Nnn {\l__sparser_state_str..} {1} {4}}\ 
		\bool_if:nTF \l__sparser_ignore_bool {\textcolor{spred}{ignored:\ }}{}
	}

}

%%%%%%%%%%%%%%%%%%%%%%%%%%%%%
% \sparser_writedata:n
%%%%%%%%%%%%%%%%%%%%%%%%%%%%%

% clists are used as internal buffers, which are handled here.
% when complete, the buffer is written to file and cleared.
\cs_new:Npn \sparser_writedata:n #1 {
	\str_case_e:nnTF \l__sparser_state_str
	{
		{comp} {
%			add line to comma seperated list
			\clist_put_right:Nx \l__sparser_data_comp_clist {#1}
%			remove all characters except digits item-wise
			\sparser_data_getdigits:n \l__sparser_data_comp_clist
%			prepend batch number
			\clist_put_left:Nx \l__sparser_data_comp_clist {{\int_use:N \l__sparser_batch_int}}
			\sparser_debug_out:n {
				\clist_use:Nn \l__sparser_data_comp_clist {,}
				\ \textcolor{sporange}{buffered}
			}
%			 set time clist equal to comp clist
			\clist_set_eq:NN \l__sparser_data_time_clist \l__sparser_data_comp_clist
%			clear list
			\clist_clear:N \l__sparser_data_comp_clist
			
		}
		{exec} {
%			if the list is empty wether no rows where affected or nocount was used
%			add zero to list then
			\clist_if_empty:NTF \l__sparser_data_exec_clist
				{\clist_put_right:Nn \l__sparser_data_exec_clist {0}}
				{}
%			add line to comma seperated list
			\clist_put_right:Nx \l__sparser_data_exec_clist {#1}
%			remove all characters except digits item-wise
			\sparser_data_getdigits:n \l__sparser_data_exec_clist
%			prepend statement number
			\clist_put_left:Nx \l__sparser_data_exec_clist {{\int_use:N \l__sparser_statement_int}}
			\sparser_debug_out:n {
				\clist_use:Nn \l__sparser_data_exec_clist {,}
				\ \textcolor{spdarkgreen}{written}
			}
%			append exec clist to times clist and write to file
			\clist_concat:NNN \l__sparser_data_time_clist \l__sparser_data_time_clist \l__sparser_data_exec_clist
%			write to file			
			\iow_now:Nx \g__sparser_write_time_iow {\clist_use:Nn \l__sparser_data_time_clist {,}}
%			reset list
			\clist_set:Nn \l__sparser_data_time_clist {0,0,0}
%			clear list
			\clist_clear:N \l__sparser_data_exec_clist
		}
		{io} {
%			store line in variable
			\tl_set:Nn \l__sparser_line_tl {#1}
%			replace ". Scan" with ", Scan" to make the table name the first item
			\tl_replace_once:Nnn \l__sparser_line_tl {.~Scan} {,~Scan}
%			add line to comma seperated list
			\clist_put_right:Nx \l__sparser_data_io_clist {\l__sparser_line_tl}
			\clist_pop:NN \l__sparser_data_io_clist \l__sparser_tablename_tl
%			remove "Table '...'" and "-Tabelle" (depending on language)
			\tl_replace_once:Nnn \l__sparser_tablename_tl {-Tabelle} {}
			\tl_replace_once:Nnn \l__sparser_tablename_tl{Table~} {}
			\tl_replace_all:Nnn \l__sparser_tablename_tl {'} {}
%			remove all characters except digits item-wise
			\sparser_data_getdigits:n \l__sparser_data_io_clist
%			prepend table name
			\clist_put_left:Nx \l__sparser_data_io_clist {{\tl_use:N \l__sparser_tablename_tl}}
%			write to file
			\iow_now:Nx \g__sparser_write_io_iow {\clist_use:Nn \l__sparser_data_io_clist {,}}
			\sparser_debug_out:n {
				\clist_map_inline:Nn \l__sparser_data_io_clist {\str_range:Nnn {\detokenize{##1}}{1}{12},}
				\ \textcolor{spdarkgreen}{written}
			}
%			clear list
			\clist_clear:N \l__sparser_data_io_clist
		}
		{rows} {
%			check if exec list is empty
			\clist_if_empty:NTF \l__sparser_data_exec_clist
			{
				\clist_put_right:Nx \l__sparser_data_exec_clist {#1}
				\sparser_data_getdigits:n \l__sparser_data_exec_clist
				\sparser_debug_out:n {
					\clist_use:Nn \l__sparser_data_exec_clist {,}
					\ \textcolor{sporange}{buffered}
				}
			}
			{}
		}
	}
	{}
	{
		\sparser_debug_out:n {#1\ \textcolor{spred}{ignored}}
	}
\sparser_debug_out:n {\\}
}

%%%%%%%%%%%%%%%%%%%%%%%%%%%%%
% \sparser_writedata:n
%%%%%%%%%%%%%%%%%%%%%%%%%%%%%

% removes all non-digits from the elements of a clist
\cs_new:Npn \sparser_data_getdigits:n #1 {
	\clist_map_inline:Nn #1 {
		\regex_replace_all:nnN {[^[,0-9]]*} {} {#1}
	}
}

%%%%%%%%%%%%%%%%%%%%%%%%%%%%%
% \sparser_file_time:nnn & \sparser_file_io:nnn
%%%%%%%%%%%%%%%%%%%%%%%%%%%%%

\cs_new:Npn \sparser_file_time:nnn #1 #2 #3 {#1/#2_time#3}
\cs_new:Npn \sparser_file_io:nnn #1 #2 #3 {#1/#2_io_b\int_use:N \l__sparser_batch_int s\int_use:N \l__sparser_statement_int#3}

%%%%%%%%%%%%%%%%%%%%%%%%%%%%%
% \sparser_close_stream:n
%%%%%%%%%%%%%%%%%%%%%%%%%%%%%

\cs_new:Npn \sparser_close_stream:n #1 {
	\str_case:nn {#1}
	{
		{io} {
%			check if exec output stream is open
			\bool_if:NTF {\l__sparser_output_io_open_bool}
				{
%					close exec output stream and set flag
					\iow_close:N \g__sparser_write_io_iow
					\bool_set_false:N \l__sparser_output_io_open_bool
					\sparser_debug_out:n {
						...closing\ \textcolor{spdarkgreen}{io}\ output\ stream\ 
						\textcolor{spdarkgreen}{\l__sparser_output_io_str}\\
					}
				}
				{}
		}
		{time} {
%			check if exec output stream is open
			\bool_if:NTF {\l__sparser_output_time_open_bool}
				{
%					close exec output stream and set flag
					\iow_close:N \g__sparser_write_time_iow
					\bool_set_false:N \l__sparser_output_time_open_bool
					\sparser_debug_out:n {
						...closing\ \textcolor{spdarkgreen}{time}\ output\ stream\ 
						\textcolor{spdarkgreen}{\l__sparser_output_time_str}\\}
				}
				{}
		}
	}
}

%%%%%%%%%%%%%%%%%%%%%%%%%%%%%
% \sparser_open_stream:n
%%%%%%%%%%%%%%%%%%%%%%%%%%%%%

\cs_new:Npn \sparser_open_stream:n #1 {
	\str_case:nn {#1}
	{
		{io} {
%				generate and store output file name
				\str_set:Nx \l__sparser_output_io_str	{
					\file_parse_full_name_apply:nN {\l__sparser_inputfile_str} \sparser_file_io:nnn
				}
%				open exec stream and set flag
				\iow_open:Nn \g__sparser_write_io_iow \l__sparser_output_io_str
				\iow_now:Nx \g__sparser_write_io_iow {Table, Scan count, logical reads, physical reads, read-ahead reads, lob logical reads, lob physical reads, lob read-ahead reads}
				\bool_set_true:N \l__sparser_output_io_open_bool
				\sparser_debug_out:n {
					opening\ \textcolor{spdarkgreen}{io}\ output\ stream\ 
					\textcolor{spdarkgreen}{\l__sparser_output_io_str}...\\
				}
		}
		{time} {
%				generate and store output file name
				\str_set:Nx \l__sparser_output_time_str	{
					\file_parse_full_name_apply:nN {\l__sparser_inputfile_str} \sparser_file_time:nnn
				}
%				open exec stream and set flag
				\iow_open:Nn \g__sparser_write_time_iow \l__sparser_output_time_str
				\iow_now:Nx \g__sparser_write_time_iow {Batch,CompCPUtime,CompElapsedtime,Statement,Rows,ExecCPUtime,ExecElapsedtime}
				\bool_set_true:N \l__sparser_output_time_open_bool
				\sparser_debug_out:n {
					opening\ \textcolor{spdarkgreen}{time}\ output\ stream\ 
					\textcolor{spdarkgreen}{\l__sparser_output_time_str}...\\
				}
		}
	}
}

%%%%%%%%%%%%%%%%%%%%%%%%%%%%%
% \sparsertime
%%%%%%%%%%%%%%%%%%%%%%%%%%%%%
\NewDocumentCommand{\sparsertime}{m O{0}} {
\DTLloadrawdb[keys={batch,compcpu,compelapsed,statement,rows,execcpu,execelapsed}]{io1}{#1}

\DTLsetnumberchars{}{,}
\DTLsumcolumn{io1}{compcpu}{\tcompcpu}%
\DTLsumcolumn{io1}{compelapsed}{\tcompelapsed}%
\DTLsumcolumn{io1}{rows}{\trows}%
\DTLsumcolumn{io1}{execcpu}{\texeccpu}%
\DTLsumcolumn{io1}{execelapsed}{\texecelapsed}%

\small{
\begin{tabularx}{\linewidth}{
>{\raggedleft\arraybackslash\hsize=0.6\hsize}X
*{2}{>{\raggedleft\arraybackslash\hsize=1.15\hsize}X}
*{2}{>{\raggedleft\arraybackslash\hsize=0.9\hsize}X}
*{2}{>{\raggedleft\arraybackslash\hsize=1.15\hsize}X}
}
\toprule
% Do the header row.
{\bfseries\sparser_dict:n{Batch}}
& \multicolumn{2}{c}{\bfseries\sparser_dict:n{Parse and compile time}}%
& {\bfseries\sparser_dict:n{Statement}}
&{\bfseries\sparser_dict:n{Rows}}
& \multicolumn{2}{c}{\bfseries\sparser_dict:n{Execution Times}}%
\\
& {\sparser_dict:n{CPU}~[ms]} & {\sparser_dict:n{elapsed}~[ms]} & & & {\sparser_dict:n{CPU}~[ms]} & {\sparser_dict:n{elapsed}~[ms]}%
% Iterate through the data.
\DTLforeach*{io1}{\varbatch=batch,\varcompcpu=compcpu,\varcompelapsed=compelapsed,\varstatement=statement,\varrows=rows,\varexeccpu=execcpu,\varexecelapsed=execelapsed}{%
\gdef\doamp{&}%
%line break and first line
\DTLiffirstrow{\\\midrule}{
	\dtlifnumeq{\varbatch}{0}{
		\dtlifnumeq{\skiprow}{0}{\\}{}
	}{\\}
}
% columns 1-3
\dtlifnumeq{\varbatch}{0}{%
	\dtlifnumeq{\skiprow}{0}{
		\doamp\doamp
	}{}%
}{%
	\num{\varbatch}
	\doamp\num{\varcompcpu}
	\doamp\num{\varcompelapsed}
}%
% calculate skiprow flag
\gdef\skiprow{#2}%
\dtlifnumeq{\skiprow}{1}{%
	\dtlifnumeq{\varrows}{0}{}{\gdef\skiprow{0}}%
	\dtlifnumeq{\varexeccpu}{0}{}{\gdef\skiprow{0}}%
	\dtlifnumeq{\varexecelapsed}{0}{}{\gdef\skiprow{0}}%
}{}%
% skiprow=0: show columns 4-7
\dtlifnumeq{\skiprow}{0}{
	\doamp\num{\varstatement}
	\doamp\num{\varrows}
	\doamp\num{\varexeccpu}
	\doamp\num{\varexecelapsed}%
}{}
}%foreach
\\\midrule
\sparser_dict:n{Total}
\doamp\num{\tcompcpu}
\doamp\num{\tcompelapsed}
\doamp
\doamp\num{\trows}
\doamp\num{\texeccpu}
\doamp\num{\texecelapsed}
\\\bottomrule%
\end{tabularx}%
}%end small
\DTLgdeletedb{io1}%
}

%%%%%%%%%%%%%%%%%%%%%%%%%%%%%
% \sparserio
%%%%%%%%%%%%%%%%%%%%%%%%%%%%%
\NewDocumentCommand{\sparserio}{o m o O{0} O{1}} {
	\IfNoValueTF {#3} 
		{\sparseriosingle {#2}[#1]}
		{\sparseriocompare {#2}{#3}[#4][#5][#1]}
}

%%%%%%%%%%%%%%%%%%%%%%%%%%%%%
% \sparseriosingle
%%%%%%%%%%%%%%%%%%%%%%%%%%%%%
\NewDocumentCommand{\sparseriosingle}{m O{0}} {
\DTLloadrawdb[keys={table,scan,logical,physical,readahead,logicalLOB,physicalLOB,readaheadLOB}]{io1}{#1}
\dtlsort{table}{io1}{\dtlwordindexcompare}

\DTLsetnumberchars{}{,}
\DTLsumcolumn{io1}{scan}{\tscan}%
\DTLsumcolumn{io1}{logical}{\tlogical}%
\DTLsumcolumn{io1}{physical}{\tphysical}%
\DTLsumcolumn{io1}{readahead}{\treadahead}%
\DTLsumcolumn{io1}{logicalLOB}{\tlogicalLOB}%
\DTLsumcolumn{io1}{physicalLOB}{\tphysicalLOB}%
\DTLsumcolumn{io1}{readaheadLOB}{\treadaheadLOB}%

\small{
\begin{tabularx}{\linewidth}{
%8 columns: \hsize is \linewidth/8 by default
%make first column wider: 1.7 + 7*0.9 = 8
>{\hsize=1.7\hsize}X
*{7}{>{\raggedleft\arraybackslash\hsize=0.9\hsize}X}
}
\toprule
% Do the header row.
{\bfseries\sparser_dict:n{Table}}%
&{\bfseries\sparser_dict:n{Scans}}%
& \multicolumn{3}{c}{\bfseries\sparser_dict:n{Reads}}%
& \multicolumn{3}{c}{\bfseries\sparser_dict:n{LOB-Reads}}%
\\%
& & {\sparser_dict:n{logical}}
& {\sparser_dict:n{physical}}
& {\sparser_dict:n{r.-ahead}}
& {\sparser_dict:n{logical}}
& {\sparser_dict:n{physical}}
& {\sparser_dict:n{r.-ahead}}%
% Iterate through the data.
\DTLforeach*{io1}{}{%
\\\DTLiffirstrow{\midrule}{}%
\gdef\doamp{\gdef\doamp{&}}%
\DTLforeachkeyinrow{\thisValue}{
\doamp\DTLifint{\thisValue}{\num{\thisValue}}{\sparser_breakanywhere:Nn{\thisValue}{#2}}
}%foreachkeyinrow
}%foreach
\\\midrule
\sparser_dict:n{Total}
\doamp\num{\tscan}
\doamp\num{\tlogical}
\doamp\num{\tphysical}
\doamp\num{\treadahead}
\doamp\num{\tlogicalLOB}
\doamp\num{\tphysicalLOB}
\doamp\num{\treadaheadLOB}
\\\bottomrule%
\end{tabularx}
}
\DTLgdeletedb{io1}
}


%%%%%%%%%%%%%%%%%%%%%%%%%%%%%
% \sparseriocompare
%%%%%%%%%%%%%%%%%%%%%%%%%%%%%
\int_new:N \l__sparser_ioSumAllA_int
\int_new:N \l__sparser_ioSumAllB_int

\NewDocumentCommand{\sparseriocompare}{m m O{0} O{1} O{0}} {
\DTLloadrawdb[keys={table,scan,logical,physical,readahead,logicalLOB,physicalLOB,readaheadLOB}]{io1}{#1}
\DTLloadrawdb[keys={table,scan,logical,physical,readahead,logicalLOB,physicalLOB,readaheadLOB}]{io2}{#2}

\dtlsort{table}{io1}{\dtlwordindexcompare}
\dtlsort{table}{io2}{\dtlwordindexcompare}

\sparser_match_tables:nn {io1}{io2}

\dtlsort{table}{io1}{\dtlwordindexcompare}
\dtlsort{table}{io2}{\dtlwordindexcompare}

\DTLsetnumberchars{}{,}
\DTLsumcolumn{io1}{scan}{\tscan}%
\DTLsumcolumn{io1}{logical}{\tlogical}%
\DTLsumcolumn{io1}{physical}{\tphysical}%
\DTLsumcolumn{io1}{readahead}{\treadahead}%
\DTLsumcolumn{io1}{logicalLOB}{\tlogicalLOB}%
\DTLsumcolumn{io1}{physicalLOB}{\tphysicalLOB}%
\DTLsumcolumn{io1}{readaheadLOB}{\treadaheadLOB}%

\DTLsumcolumn{io2}{scan}{\tscanb}%
\DTLsumcolumn{io2}{logical}{\tlogicalb}%
\DTLsumcolumn{io2}{physical}{\tphysicalb}%
\DTLsumcolumn{io2}{readahead}{\treadaheadb}%
\DTLsumcolumn{io2}{logicalLOB}{\tlogicalLOBb}%
\DTLsumcolumn{io2}{physicalLOB}{\tphysicalLOBb}%
\DTLsumcolumn{io2}{readaheadLOB}{\treadaheadLOBb}%

\int_set:Nn \l__sparser_ioSumAllA_int  {\tscan+\tlogical+\tphysical+\treadahead+\tlogicalLOB+\tphysicalLOB+\treadaheadLOB}
\int_set:Nn \l__sparser_ioSumAllB_int  {\tscanb+\tlogicalb+\tphysicalb+\treadaheadb+\tlogicalLOBb+\tphysicalLOBb+\treadaheadLOBb}

\small{
\begin{tabularx}{\linewidth}{
%8 columns: \hsize is \linewidth/8 by default
%make first column wider: 1.7 + 7*0.9 = 8
>{\hsize=1.7\hsize}X
*{7}{>{\raggedleft\arraybackslash\hsize=0.9\hsize}X}
}
\toprule
% Do the header row.
{\bfseries\sparser_dict:n{Table}}
&{\bfseries\sparser_dict:n{Scans}}
& \multicolumn{3}{c}{\bfseries\sparser_dict:n{Reads}}%
& \multicolumn{3}{c}{\bfseries\sparser_dict:n{LOB-Reads}}%
\\
& & {\sparser_dict:n{logical}}
& {\sparser_dict:n{physical}}
& {\sparser_dict:n{r.-ahead}}
& {\sparser_dict:n{logical}}
& {\sparser_dict:n{physical}}
& {\sparser_dict:n{r.-ahead}}
% Iterate through the data.
\gdef\isfirstrow{1}%
\DTLforeach*{io1}{\vartable=table,\varscan=scan,\varlogical=logical,\varphysical=physical,\varreadahead=readahead,\varlogicalLOB=logicalLOB,\varphysicalLOB=physicalLOB,\varreadaheadLOB=readaheadLOB}{%
\DTLassign{io2}{\DTLcurrentindex}{\vartableb=table,\varscanb=scan,\varlogicalb=logical,\varphysicalb=physical,\varreadaheadb=readahead,\varlogicalLOBb=logicalLOB,\varphysicalLOBb=physicalLOB,\varreadaheadLOBb=readaheadLOB}%
\gdef\doamp{&}%
\gdef\skiprow{#3}%
\dtlifnumeq{\skiprow}{1}{%
	\dtlifnumeq{\varscanb}{\varscan}{}{\gdef\skiprow{0}}%
	\dtlifnumeq{\varlogicalb}{\varlogical}{}{\gdef\skiprow{0}}%
	\dtlifnumeq{\varphysicalb}{\varphysical}{}{\gdef\skiprow{0}}%
	\dtlifnumeq{\varreadaheadb}{\varreadahead}{}{\gdef\skiprow{0}}%
	\dtlifnumeq{\varlogicalLOBb}{\varlogicalLOB}{}{\gdef\skiprow{0}}%
	\dtlifnumeq{\varphysicalLOBb}{\varphysicalLOB}{}{\gdef\skiprow{0}}%
	\dtlifnumeq{\varreadaheadLOBb}{\varreadaheadLOB}{}{\gdef\skiprow{0}}%
}{}%
\dtlifnumeq{\skiprow}{1}{}{%
\dtlifnumeq{\isfirstrow}{1}{\\\midrule\gdef\isfirstrow{0}}{\\\rule{0pt}{2.5ex}}%
\multirow[t]{2}{\hsize}{\sparser_breakanywhere:Nn{\vartable}{#5}}
\doamp\num{\varscan}
\doamp\num{\varlogical}
\doamp\num{\varphysical}
\doamp\num{\varreadahead}
\doamp\num{\varlogicalLOB}
\doamp\num{\varphysicalLOB}
\doamp\num{\varreadaheadLOB}
\\
\doamp\sparser_colorscale:nnnnn{\varscan}{\varscanb}{\l__sparser_ioSumAllA_int}{\l__sparser_ioSumAllB_int}{#4}\num{\varscanb}
\doamp\sparser_colorscale:nnnnn{\varlogical}{\varlogicalb}{\l__sparser_ioSumAllA_int}{\l__sparser_ioSumAllB_int}{#4}\num{\varlogicalb}
\doamp\sparser_colorscale:nnnnn{\varphysical}{\varphysicalb}{\l__sparser_ioSumAllA_int}{\l__sparser_ioSumAllB_int}{#4}\num{\varphysicalb}
\doamp\sparser_colorscale:nnnnn{\varreadahead}{\varreadaheadb}{\l__sparser_ioSumAllA_int}{\l__sparser_ioSumAllB_int}{#4}\num{\varreadaheadb}
\doamp\sparser_colorscale:nnnnn{\varlogicalLOB}{\varlogicalLOBb}{\l__sparser_ioSumAllA_int}{\l__sparser_ioSumAllB_int}{#4}\num{\varlogicalLOBb}
\doamp\sparser_colorscale:nnnnn{\varphysicalLOB}{\varphysicalLOBb}{\l__sparser_ioSumAllA_int}{\l__sparser_ioSumAllB_int}{#4}\num{\varphysicalLOBb}
\doamp\sparser_colorscale:nnnnn{\varreadaheadLOB}{\varreadaheadLOBb}{\l__sparser_ioSumAllA_int}{\l__sparser_ioSumAllB_int}{#4}\num{\varreadaheadLOBb}%
}}%end foreach*
\\\midrule
\sparser_dict:n{Total}
\doamp\num{\tscan}
\doamp\num{\tlogical}
\doamp\num{\tphysical}
\doamp\num{\treadahead}
\doamp\num{\tlogicalLOB}
\doamp\num{\tphysicalLOB}
\doamp\num{\treadaheadLOB}
\\
\doamp\sparser_colorscale:nnnnn{\tscan}{\tscanb}{\l__sparser_ioSumAllA_int}{\l__sparser_ioSumAllB_int}{#4}\num{\tscanb}
\doamp\sparser_colorscale:nnnnn{\tlogical}{\tlogicalb}{\l__sparser_ioSumAllA_int}{\l__sparser_ioSumAllB_int}{#4}\num{\tlogicalb}
\doamp\sparser_colorscale:nnnnn{\tphysical}{\tphysicalb}{\l__sparser_ioSumAllA_int}{\l__sparser_ioSumAllB_int}{#4}\num{\tphysicalb}
\doamp\sparser_colorscale:nnnnn{\treadahead}{\treadaheadb}{\l__sparser_ioSumAllA_int}{\l__sparser_ioSumAllB_int}{#4}\num{\treadaheadb}
\doamp\sparser_colorscale:nnnnn{\tlogicalLOB}{\tlogicalLOBb}{\l__sparser_ioSumAllA_int}{\l__sparser_ioSumAllB_int}{#4}\num{\tlogicalLOBb}
\doamp\sparser_colorscale:nnnnn{\tphysicalLOB}{\tphysicalLOBb}{\l__sparser_ioSumAllA_int}{\l__sparser_ioSumAllB_int}{#4}\num{\tphysicalLOBb}
\doamp\sparser_colorscale:nnnnn{\treadaheadLOB}{\treadaheadLOBb}{\l__sparser_ioSumAllA_int}{\l__sparser_ioSumAllB_int}{#4}\num{\treadaheadLOBb}%
\\\bottomrule%
\end{tabularx}%
}%

\DTLgdeletedb{io1}%
\DTLgdeletedb{io2}%
}

% variables for \sparser_match_tables:nn

%sequences with table names
\seq_new:N \l__sparser_tablesA_seq
\seq_new:N \l__sparser_tablesB_seq
%only for debug: table names that are added
\seq_new:N \l__sparser_tablesAnew_seq
\seq_new:N \l__sparser_tablesBnew_seq
%string variables to store a single table name
\str_new:N \l__sparser_tableNameA_str
\str_new:N \l__sparser_tableNameB_str
%int representing the relative lexicographical order of two string
\int_new:N \l__sparser_tableCompare_int
%loop condition
\bool_new:N \l__sparser_loop_bool

%%%%%%%%%%%%%%%%%%%%%%%%%%%%%
% \sparser_match_tables:nn
%%%%%%%%%%%%%%%%%%%%%%%%%%%%%

% compares the "table" comlumns of two databases and matches them with each other by adding rows with zeros
% assumes the databases to be ordered by the "table" column
\cs_new:Npn \sparser_match_tables:nn #1 #2
{
	%set debug out
	\bool_set_false:N \l__sparser_debug_bool
	
	%store table names of db1 in sequence A
	\DTLforeach*{#1}{\l__sparser_tableNameA_str=table} {
		\seq_put_right:Nx \l__sparser_tablesA_seq {\l__sparser_tableNameA_str}
	}
	
	%store table names of db1 in sequence B
	\DTLforeach*{#2}{\l__sparser_tableNameB_str=table} {
		\seq_put_right:Nx \l__sparser_tablesB_seq {\l__sparser_tableNameB_str}
	}
	
	%loop condition: loop while seq A or seq B ist not empty
	\bool_set:Nn \l__sparser_loop_bool {
		!\seq_if_empty_p:N \l__sparser_tablesA_seq || !\seq_if_empty_p:N \l__sparser_tablesB_seq
	}
	
	%set flag to make datatool expand the tablename variable when adding a new value
	\dtlexpandnewvalue
	
	%loop: both sequences act as stacks
	%with every iteration the first element of at least one of the stacks is popped
	%this is decided by a comparison of the two table names
	\bool_do_while:Nn \l__sparser_loop_bool
	{
		%debug out sequences, display "empty" if sequence is empty
		%A
		\seq_if_empty:NTF \l__sparser_tablesA_seq
		{
			\sparser_debug_out:n {A:\ \textcolor{spred}{empty}\\}
		}{
			\sparser_debug_out:n {A:\ \seq_use:Nn \l__sparser_tablesA_seq {,}\\}
		}
		
		%B
		\seq_if_empty:NTF \l__sparser_tablesB_seq
		{
			\sparser_debug_out:n {B:\ \textcolor{spred}{empty}\\}
		}{
			\sparser_debug_out:n {B:\ \seq_use:Nn \l__sparser_tablesB_seq {,}\\}
		}
		
		% store first element of both stacks
		\seq_get:NN \l__sparser_tablesA_seq \l__sparser_tableNameA_str
		\seq_get:NN \l__sparser_tablesB_seq \l__sparser_tableNameB_str
		
		% compare both elements and set \l__sparser_tableCompare_int accordingly (-1, 0, 1)
		\seq_if_empty:NTF \l__sparser_tablesA_seq 
		{
			%seq A is empty
			\int_set:Nn \l__sparser_tableCompare_int {1}
			% no check on B needed, since at least one sequence is always not empty (loop condition)
		}
		{
			\seq_if_empty:NTF \l__sparser_tablesB_seq
			{
				%seq B is empty
				\int_set:Nn \l__sparser_tableCompare_int {-1}
			}
			{
				% both tables not empty
				\str_if_eq:NNTF \l__sparser_tableNameA_str \l__sparser_tableNameB_str
				{
					%both table names are equal
					\int_set:Nn \l__sparser_tableCompare_int {0}
				}
				{
					% check if table name B is also in sequence A
					\seq_if_in:NxTF \l__sparser_tablesA_seq \l__sparser_tableNameB_str
					{
						%yes => table name A is not in sequence B, because the sequences are ordered!
						\int_set:Nn \l__sparser_tableCompare_int {-1}
					}
					{
						%no => table name B is not in sequence A
						\int_set:Nn \l__sparser_tableCompare_int {1}					
					}
				}
			}
		}

		% action depending on the result of the comparison				
		\int_case:nn {\l__sparser_tableCompare_int}
		{
			%0: both table names are equal and can be discarded
			{0} {
				\seq_pop:NN \l__sparser_tablesA_seq \l__sparser_tableNameA_str
				\seq_pop:NN \l__sparser_tablesB_seq \l__sparser_tableNameB_str
				\sparser_debug_out:n {pop A: \str_use:N \l__sparser_tableNameA_str\\}
				\sparser_debug_out:n {pop B: \str_use:N \l__sparser_tableNameB_str\\}
			}
			%1: table name B is not in sequence A and thus has to be added
			{1} {
				% table name B can then removed from the stack
				\seq_pop:NN \l__sparser_tablesB_seq \l__sparser_tableNameB_str
				% add row to database A
				\DTLnewrow{#1}
				\DTLnewdbentry{#1}{table}{\l__sparser_tableNameB_str}
				\DTLnewdbentry{#1}{scan}{0}
				\DTLnewdbentry{#1}{logical}{0}
				\DTLnewdbentry{#1}{physical}{0}
				\DTLnewdbentry{#1}{readahead}{0}
				\DTLnewdbentry{#1}{logicalLOB}{0}
				\DTLnewdbentry{#1}{physicalLOB}{0}
				\DTLnewdbentry{#1}{readaheadLOB}{0}	
				%debug out
				\sparser_debug_out:n {pop B: \str_use:N \l__sparser_tableNameB_str\\}
				\seq_put_right:Nx \l__sparser_tablesAnew_seq \l__sparser_tableNameB_str
			}
			%-1: table name A is not in sequence B and thus has to be added
			{-1} {
				% table name A can then removed from the stack
				\seq_pop:NN \l__sparser_tablesA_seq \l__sparser_tableNameA_str
				% add row to database B
				\DTLnewrow{#2}
				\DTLnewdbentry{#2}{table}{\l__sparser_tableNameA_str}
				\DTLnewdbentry{#2}{scan}{0}
				\DTLnewdbentry{#2}{logical}{0}
				\DTLnewdbentry{#2}{physical}{0}
				\DTLnewdbentry{#2}{readahead}{0}
				\DTLnewdbentry{#2}{logicalLOB}{0}
				\DTLnewdbentry{#2}{physicalLOB}{0}
				\DTLnewdbentry{#2}{readaheadLOB}{0}
				%debug out
				\sparser_debug_out:n {pop A: \str_use:N \l__sparser_tableNameA_str\\}
				\seq_put_right:Nx \l__sparser_tablesBnew_seq \l__sparser_tableNameA_str
			}
		}

		%debug out sequence of newly added table names
		\seq_if_empty:NTF \l__sparser_tablesAnew_seq
		{
			\sparser_debug_out:n {A new:\ \textcolor{spred}{empty}\\}
		}{
			\sparser_debug_out:n {A new:\ \seq_use:Nn \l__sparser_tablesAnew_seq {,}\\}
		}
		
		\seq_if_empty:NTF \l__sparser_tablesBnew_seq
		{
			\sparser_debug_out:n {B new:\ \textcolor{spred}{empty}\\}
		}{
			\sparser_debug_out:n {B new:\ \seq_use:Nn \l__sparser_tablesBnew_seq {,}\\}
		}
				
		%update loop condition
		\bool_set:Nn \l__sparser_loop_bool {
			!\seq_if_empty_p:N \l__sparser_tablesA_seq || !\seq_if_empty_p:N \l__sparser_tablesB_seq
		}
		%debug out
		\sparser_debug_out:n {--------\\}
	}
	
	%clear sequences and reset flag
	\seq_clear:N \l__sparser_tablesAnew_seq
	\seq_clear:N \l__sparser_tablesBnew_seq
	\dtlnoexpandnewvalue
}

%%%%%%%%%%%%%%%%%%%%%%%%%%%%%
% \sparser_breakanywhere:Nn
%%%%%%%%%%%%%%%%%%%%%%%%%%%%%
\cs_new:Npn \sparser_breakanywhere:Nn #1 #2 {
	\int_compare:nTF {#2 = 0}
	{
		\str_set:Nx \l__sparser_tableNameA_str {#1}
	}{
		\int_compare:nTF {\str_count:N #1>#2}{
			\int_compare:nTF {\str_count:N #2>4}{
				\str_set:Nx \l__sparser_tableNameA_str {\str_range:Nnn #1 {1}{#2-4}}
			} {
				\str_set:Nx \l__sparser_tableNameA_str {\str_range:Nnn #1 {1}{#2}}
			}
		}{
			\str_set:Nx \l__sparser_tableNameA_str {\str_range:Nnn #1 {1}{#2}}
		}
	}
	\str_map_inline:Nn  \l__sparser_tableNameA_str {
		##1\allowbreak
	}
	\int_compare:nTF {#2 = 0}{}{\int_compare:nTF {\str_count:N #1>#2}{\int_compare:nTF {\str_count:N #2>4}{...}{}}{}}
}

%%%%%%%%%%%%%%%%%%%%%%%%%%%%%
% \sparser_colorscale:nnnnn
%%%%%%%%%%%%%%%%%%%%%%%%%%%%%
\cs_new:Npn \sparser_colorscale:nnnnn #1 #2 #3 #4 #5{
	\int_compare:nTF {#5 = 0} {} {
		\int_compare:nTF {#1 = #2} {} {
			\int_compare:nNnTF {\int_max:nn{#3}{#4}} = {0} {} {
			% (|#2-#1|)/max(#3,#4) in [0,1]
			% x^(1/#5): [0,1] -> [0,1] scaling to make small values more visible
				\cellcolor{\int_compare:nNnTF {#1} < {#2} {spred!}{spdarkgreen!}
				\fp_to_int:n{\fp_eval:n{100*(\fp_abs:n{(#1-#2)/\fp_max:nn{#3}{#4}})**(1/#5)}}}
			}
		}
	}
}

%%%%%%%%%%%%%%%%%%%%%%%%%%%%%
% \sparser_dict:n 
%%%%%%%%%%%%%%%%%%%%%%%%%%%%%

\cs_new:Npn \sparser_dict:n #1 {
	\str_case:nn {#1} {
		{Table} {
			\str_case_e:nnTF {\languagename} {
				{english} {Table}
				{german} {Tabelle}				
				{ngerman} {Tabelle}
			}
			{}
			{#1}				
		}
		{Scans} {
			\str_case_e:nnTF {\languagename} {
				{english} {Scans}
				{german} {Scans}				
				{ngerman} {Scans}
			}
			{}
			{#1}				
		}
		{Reads} {
			\str_case_e:nnTF {\languagename} {
				{english} {Reads}
				{german} {Lesevorgänge}				
				{ngerman} {Lesevorgänge}
			}
			{}
			{#1}				
		}
		{LOB-Reads} {
			\str_case_e:nnTF {\languagename} {
				{english} {LOB-Reads}
				{german} {LOB-Lesevorgänge}				
				{ngerman} {LOB-Lesevorgänge}
			}
			{}
			{#1}				
		}
		{logical} {
			\str_case_e:nnTF {\languagename} {
				{english} {logical}
				{german} {logisch}				
				{ngerman} {logisch}
			}
			{}
			{#1}				
		}
		{physical} {
			\str_case_e:nnTF {\languagename} {
				{english} {physical}
				{german} {phys.}				
				{ngerman} {phys.}
			}
			{}
			{#1}				
		}
		{r.-ahead} {
			\str_case_e:nnTF {\languagename} {
				{english} {r.-ahead}
				{german} {r.-ahead}				
				{ngerman} {r.-ahead}
			}
			{}
			{#1}				
		}
		{Batch} {
			\str_case_e:nnTF {\languagename} {
				{english} {Batch}
				{german} {Batch}				
				{ngerman} {Batch}
			}
			{}
			{#1}				
		}
		{Parse and compile time} {
			\str_case_e:nnTF {\languagename} {
				{english} {Parse~and~compile~time}
				{german} {Analyse-~und~Kompilierzeit}				
				{ngerman} {Analyse-~und~Kompilierzeit}
			}
			{}
			{#1}				
		}
		{Statement} {
			\str_case_e:nnTF {\languagename} {
				{english} {Stmt}
				{german} {Statement}				
				{ngerman} {Statement}
			}
			{}
			{#1}				
		}
		{Rows} {
			\str_case_e:nnTF {\languagename} {
				{english} {Rows}
				{german} {Reihen}				
				{ngerman} {Reihen}
			}
			{}
			{#1}				
		}
		{Execution Times} {
			\str_case_e:nnTF {\languagename} {
				{english} {Execution~Times}
				{german} {Ausführungszeiten}				
				{ngerman} {Ausführungszeiten}
			}
			{}
			{#1}				
		}
		{CPU} {
			\str_case_e:nnTF {\languagename} {
				{english} {CPU}
				{german} {CPU-Zeit}				
				{ngerman} {CPU-Zeit}
			}
			{}
			{#1}				
		}
		{elapsed} {
			\str_case_e:nnTF {\languagename} {
				{english} {elapsed}
				{german} {Echtzeit}				
				{ngerman} {Echtzeit}
			}
			{}
			{#1}				
		}
		{Total} {
			\str_case_e:nnTF {\languagename} {
				{english} {Total}
				{german} {Summe}				
				{ngerman} {Summe}
			}
			{}
			{#1}				
		}
	}
}

\ExplSyntaxOff